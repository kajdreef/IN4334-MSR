%!Tex root=../main.tex

\section{Theory}
\label{sec:prob}

Nowadays, most software artifacts are developed by multiple developers in a
distributed environment. In such context, it has become increasingly important
to understand how developers interact with source code to deliver a better
product\textemdash should developers own a software artifact or should multiple
developers cooperate to develop a piece of code segment? Many software metrics
have been proposed to quantify the effect of distributed development on code
quality, including experience~\cite{Rahman:blame}, socio-technical
congruence~\cite{sarma09ICSE} \etc 

The primary focus of this study is to investigate the relationship between code
ownership and software quality. From a high-level perspective, \emph{code
ownership} describes whether the \textit{responsibility} for a certain software
artifact is spread around many developers or is mostly limited to a single
person, who can be considered its ``owner.'' It can also be interpreted as a
measure of the \textit{expertise} of a developer with respect to the code
artifact~\cite{bird:original}.
%
Practically, ownership can be defined as the proportion of code contributions
of the developers to a source code artifact over a particular period (\eg
number of commits, number of added lines, \etc) \cite{Greiler:replication}.

\subsection{Previous studies}

A number of prior studies focused on code ownership and its relationship with
software quality.
% Bird, don't touch my code
Bird \etal first examined this topic and defined ownership in terms of the
\emph{proportion of commits} on software artifacts. In particular, given a
developer and a software component, her ownership is computed as ``the ratio of
number of commits that [the considered developer] has made relative to the
total number of commits for that component''~\cite{bird:original}. They also
introduced the notion of \emph{minor contributor} (defined as each developer
who has less than 5\% ownership) and \emph{major contributor} (each developer
with more than 5\% ownership). Bird \etal put in relation these metrics with
the post-release defects in binaries (a compilation of several source code
files) from Microsoft Windows Vista and Windows 7; they found that binaries
either (1) without a well-defined owner or (2) with a large number of minor
contributors are more likely to be defective.
% Focault, code ownership in OSS
Focault \etal~\cite{Foucault:oss} replicated the above study focusing on
open-source systems, as opposed to the proprietary Microsoft Windows.


on seven
open-source projects. They used the same granularity for the metrics and the
same threshold to distinguish minor and major contributors while changing only
the code artifacts on which the study was focused (Java files and packages).
The outcome is contrasting with the previous results: it shows no strong
correlation between ownership and defects, but it states that it is more
significant when the metrics are computed on more coarse-grained artifacts.


% previous studies
Previous research investigated the relationship of ``ownership'' with software
quality and found that, in general, a high degree of ownership relates to fewer
software defects~\cite{bird:original}.

These studies are typically
conducted at release granularity, 
\ie the metrics are computed on the code artifacts of a specific software
release and then correlated with the presence of defects in it. 
One problem with such approach is that the metrics are not extracted when the
defects are introduced, but later, so they do not capture the state of the code
when it becomes defective. Developers introduce bugs while committing changes
to the code base, and the effect of their ownership of respective software
artifact may play a pivotal role to assess the quality of the commit.  Thus, in
this paper, we first investigate:

\RQ{rq1}{Is a developer with lower code ownership more likely to introduce a
buggy commit?}

There is no fixed definition of {\em ownership} in the literature. While Bird
\etal define ownership~\cite{bird:original} of a contributor as ``the ratio of
number of commits that the contributor has made relative to the total
number of commits for that component.'', Rahman et al.~\cite{Rahman:blame}
describes ownership of a contributor as a number of line written by the
contributor \wrt the total number of lines in the studied artifact. To
understand the impact of different granularity on Code ownership metric, next
we investigate: 

\RQ{rq2}{Which ownership granularity (line-based or commit-based) is a better
predictor of bug-inducing commits?}

Finally, number of contributors contributed to a software artifact up-to a
given commit may also affect the code quality. Bird \etal defined
\textit{Minor} and \textit{Major} contributors of a software artifact as the
number of contributors with a proportion of ownership that is below and above a
certain threshold respectively. As artifacts they considered the software
binaries of a Microsoft Windows release and as variable to measure the
proportions of ownership on every artifact they used the number of commits that
changed it before the release, with a 5\% threshold to identify minor and major
contributors.
These metrics have been then reused and revisited in further
studies~\cite{Foucault:oss, Greiler:replication}, targeting different projects
(Microsoft and OSS), different kind of artifacts (source files, source code
folders and Java packages) and varying the ownership threshold (5\%, 20\% and
50\%). However, none of the previous works did an explicit analysis on the
impact of this ownership threshold on defect proneness. It leads us to ask the
following question:

\RQ{rq3}{How does ownership threshold per commit impact the accuracy of defect
prediction?}
